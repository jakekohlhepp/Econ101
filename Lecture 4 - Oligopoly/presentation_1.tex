%%%%%%%%%%%%%%%%%%%%%%%%%%%%%%%%%%%%%%%%%
% Beamer Presentation
% LaTeX Template
% Version 1.0 (10/11/12)
%
% This template has been downloaded from:
% http://www.LaTeXTemplates.com
%
% License:
% CC BY-NC-SA 3.0 (http://creativecommons.org/licenses/by-nc-sa/3.0/)
%
%%%%%%%%%%%%%%%%%%%%%%%%%%%%%%%%%%%%%%%%%

%----------------------------------------------------------------------------------------
%	PACKAGES AND THEMES
%----------------------------------------------------------------------------------------

\documentclass[aspectratio=169]{beamer}
\usepackage[utf8]{inputenc}
\usepackage{booktabs}
\usepackage{graphicx}
\usepackage{array}
\usepackage{caption}
\usepackage{threeparttable}
\usepackage{lscape}
\usepackage{import}
\usepackage{amsmath}
\usepackage{csvsimple}
\usepackage{siunitx}
\usepackage{subfigure}
\usepackage{filecontents}
\newenvironment{wideitemize}{\itemize\addtolength{\itemsep}{10pt}}{\enditemize}
\usepackage{appendixnumberbeamer}
\usepackage{float}
\usepackage{amsmath}  
\usepackage{tikz,pgfplots}
\usepackage{tkz-fct}
\usepackage{amsthm}
\pgfplotsset{compat=1.10}
\usepgfplotslibrary{fillbetween}
\newcommand{\vertLineFromPoint}[1]{
  \draw[dashed] 
  (#1) -- (#1|-{rel axis cs:0,0})
}
\newcommand{\horLineFromPoint}[1]{
  \draw[dashed] 
  (#1) -- (#1-|{rel axis cs:0,0})
}
\mode<presentation> {
\AtBeginSection[]
{
    \begin{frame}
        \frametitle{Table of Contents}
        \tableofcontents[currentsection]
    \end{frame}
}
% The Beamer class comes with a number of default slide themes
% which change the colors and layouts of slides. Below this is a list
% of all the themes, uncomment each in turn to see what they look like.

%\usetheme{default}
%\usetheme{AnnArbor}
%\usetheme{Antibes} -
%\usetheme{Bergen}
%\usetheme{Berkeley}
%\usetheme{Berlin}
\usetheme{Boadilla}
%\usetheme{CambridgeUS}
%\usetheme{Copenhagen} -
%\usetheme{Darmstadt}
%\usetheme{Dresden}
%\usetheme{Frankfurt}
%\usetheme{Goettingen}
%\usetheme{Hannover}
%\usetheme{Ilmenau}
%\usetheme{JuanLesPins}
%\usetheme{Luebeck}
%\usetheme{Madrid}
%\usetheme{Malmoe}
%\usetheme{Marburg}
%\usetheme{Montpellier}
%\usetheme{PaloAlto}
%\usetheme{Pittsburgh}
%\usetheme{Rochester} -
%\usetheme{Singapore}
%\usetheme{Szeged}
%\usetheme{Warsaw}

% As well as themes, the Beamer class has a number of color themes
% for any slide theme. Uncomment each of these in turn to see how it
% changes the colors of your current slide theme.

%\usecolortheme{albatross}
%\usecolortheme{beaver}
%\usecolortheme{beetle}
%\usecolortheme{crane}
%\usecolortheme{dolphin}
%\usecolortheme{dove}
%\usecolortheme{fly}
%\usecolortheme{lily}
%\usecolortheme{orchid}
%\usecolortheme{rose}
%\usecolortheme{seagull}
%\usecolortheme{seahorse}
%\usecolortheme{whale}
%\usecolortheme{wolverine}

%\setbeamertemplate{footline} % To remove the footer line in all slides uncomment this line
\setbeamertemplate{footline}[frame number] % To replace the footer line in all slides with a simple slide count uncomment this line
\setbeamertemplate{theorems}[numbered]
\setbeamertemplate{navigation symbols}{} % To remove the navigation symbols from the bottom of all slides uncomment this line
}
\setbeamertemplate{caption}{\raggedright\insertcaption\par}
  \setbeamertemplate{enumerate items}[default]
\usepackage{graphicx} % Allows including images
\usepackage{booktabs} % Allows the use of \toprule, \midrule and \bottomrule in tables
%\usepackage {tikz}
\newtheorem*{theorem*}{Theorem}
\newtheorem*{lemma*}{Lemma}
\newtheorem*{proposition}{Proposition}
\newtheorem*{corollary*}{Corollary}
\newtheorem*{definition*}{Definition}
\DeclareMathOperator*{\argmin}{arg\,min}
\newtheorem*{assumption}{Assumption}
\usetikzlibrary {positioning}
%\usepackage {xcolor}

%----------------------------------------------------------------------------------------
%	TITLE PAGE
%----------------------------------------------------------------------------------------

\title[Oligopoly]{Lecture 4: Oligopoly} % The short title appears at the bottom of every slide, the full title is only on the title page

\author{Jacob Kohlhepp} % Your name
\institute[UCLA] % Your institution as it will appear on the bottom of every slide, may be shorthand to save space
{
Econ 101 \\ % Your institution for the title page
\medskip
}
\date{\today} % Date, can be changed to a custom date

\begin{document}

\begin{frame}
\titlepage % Print the title page as the first slide
\end{frame}

\begin{frame}{Introduction}
\begin{wideitemize}
    \item We discussed monopoly, which is when one firm has full supply-side market power.
    \item In Econ 11 we discussed perfect competition, which is where all firms have 0 market power.
    \item Now we discuss the intermediate case: when a handful of firms have some but not full supply-side market power.
    \item We will draw on our new tools from static game theory to solve games by finding Nash Equilibria in continuous strategies.
\end{wideitemize}

\end{frame}

\begin{frame}{What is Oligopoly?}
\begin{definition}
    An oligopoly is a market with relatively few firms but more than one.\footnote{Source: N\&S Chapter 15}
\end{definition}
\begin{wideitemize}
    \item In other words, not a monopoly but not perfect competition.
    \item This is more interesting than monopoly, because there is \textbf{strategic interaction}.
    \item Firms can impact the price, but not fully. They also impose an externality on the profit of other firms.
    \item \textbf{Example:} Mass media (Disney, Comcast, Viacom, News Corp), Smartphone software (Android, Apple iOS), automakers, airlines.
    \item A whole sub-field of economics is devoted to the study of strategic firm interaction: \textbf{industrial organization.}
\end{wideitemize}
\end{frame}

\begin{frame}{Road Map}
This Class:
\begin{wideitemize}
    \item Duopoly with price competition (Bertrand)
    \item Duopoly with quantity competition (Cournot)
\end{wideitemize}
Next Class:
\begin{wideitemize}
    \item Duopoly with Spatial Competition/Product Differentiation
\end{wideitemize}
In a few lectures:
\begin{wideitemize}
    \item Repeated Duopoly (Tacit Collusion)
\end{wideitemize}

    
\end{frame}
\begin{frame}{Duopoly with Price Competition (Bertrand)}
Like in all game theory problems, we layout the game:
\begin{wideitemize}
    \item \textbf{Players.} Two identical firms, numbered $1,2$.
    \item \textbf{Actions.} Firms choose prices \textit{continuously}: $0\leq p_i < \infty$
    \item \textbf{Payoffs.} 
    \begin{enumerate}
        \item Market demand is given by $D(p)$, which we assume slopes down.
        \item When firms set the same price demand is split evenly.
        \item When prices are different the lower price gets all demand.
        \item We can write this mathematically this way:
        \[D_i(p_i, p_{-i}) = \begin{cases} 0 \text{ if } p_i>p_{-i}\\
        \frac{1}{2}D(p_i) \text{ if } p_i=p_{-i}\\
        D(p_i) \text{ if } p_i<p_{-i}
        \end{cases}\]
        \item Marginal cost is constant and equal to $c$ (so fixed per unit cost of production).
    \end{enumerate}
\end{wideitemize}

    
\end{frame}

\begin{frame}{Solving Bertrand: Deriving Profit}

\[\Pi_i(p_i, p_{-i}) = \begin{cases} 0 \text{ if } p_i>p_{-i}\\
        \frac{1}{2}D(p_i) [p_i-c] \text{ if } p_i=p_{-i}\\
        D(p_i) [p_i-c] \text{ if } p_i<p_{-i}
        \end{cases}\]
    
\end{frame}

\begin{frame}{Solving Bertrand: Nash Equilibrium}

\begin{wideitemize}
    \item If you read N\&S Ch. 15, it almost seems like the authors guess the equilibrium.
    \item This is no mistake: sometimes it is easier to guess an equilibrium and then verify it satisfies the NE conditions.
    \item So we guess: $p_1=p_2=c$ is the only equilibrium.
    \item To prove this, we need to first show it is an NE. Then we need to show there are no other NE.
\end{wideitemize}

\end{frame}

\begin{frame}{Solving Bertrand: Nash Equilibrium}
First, we show it is an NE.
\begin{wideitemize}
    \item Recall the definition of an NE: every player must have no profitable deviation.
    \item To prove this is an NE, we just need to check that given the other player is playing $p_{-i}=c$, player $i$ does not gain by playing something other than $p_i=c$.
    \item First note that when $p_1=p_2=c$ profit is 0 because average cost is equal to average revenue.
    \item Suppose one player deviates to $p_i<c$ while other stays. Does the deviator gain?
    \item No: the deviator gets the full market demand, but now price is less than cost, so deviator profit is negative.
    \item What if one firm deviates to $p_i>c$? Do they gain?
    \item No: the deviator loses all demand to the other player, and makes 0 profit (the same as not deviating).
    \item Thus $p-1=p_2=c$ is an NE.
\end{wideitemize}
\end{frame}

\begin{frame}{Solving Bertrand: Nash Equilibrium}
Second, show it is the unique NE.
\begin{wideitemize}
    \item Suppose there is another NE (proof by contradiction).
    \item For clarity, just assume that the low price firm is 1: $p_1\leq p_2$.
    \item \textbf{Case 1}: $c>p_1$. In this case, firm 1 is making negative profit. This cannot be an NE because the firm could just set $p_1=c$ and at least make 0 profit.
    \item \textbf{Case 2}: $c<p_1$. Now firm 1 is making positive profit. But firm 2 is making 0 profit, and firm 2 could deviate to a price between c and $p_1$ and make positive profits.
    \item \textbf{Case 3}: $c=p_1<p_2$: Now firm 1 earns 0 profit as does firm 2. But firm 1 could slightly raise price and make positive profit.
    \item Therefore, $p_1=p_2=c$ is the unique Nash Equilibrium!
\end{wideitemize}
\end{frame}

\begin{frame}{Interpreting Bertrand}

\begin{wideitemize}
    \item What is interesting about the solution $p_1=p_2=c$?\pause
    \item It is exactly the perfect competition outcome!
    \item This is the Bertrand paradox: two firms is enough to achieve perfectly competitive prices.
    \item On the one hand this is general: we did not specify demand, and this is true even with more firms.
    \item On the other hand it is knife-edge (sensitive). It falls apart when:
    \begin{enumerate}
        \item There is product differentiation (next class).
        \item Prices are discrete (see problem set).
        \item We switch to quantity competition (next slide).
    \end{enumerate}
\end{wideitemize}
    
\end{frame}

\begin{frame}{Duopoly with quantity competition (Cournot)}
\begin{wideitemize}
    \item \textbf{Players.} $2$ firms.
    \item \textbf{Actions.} Firms choose quantities \textit{continuously}: $0\leq q_i < \infty$
    \item \textbf{Payoffs.} 
    \begin{enumerate}
        \item Total market quantity is $Q= \sum_i^2q_i$.
        \item Inverse demand (price as a function of quantity) is based on total market quantity $P(Q)$.
        \item Cost of production given by $C_i(q_i)$. Often we will have identical costs so $C_i(q_i)=C(q_i)$.
        \item Profit will be revenue less costs, like in the monopoly problem.
    \end{enumerate}
    \item Exercise: Write down the payoff for firm $i$ from producing quantity $q_i$.\pause
    \[P(Q)q_i-C_i(q_i)\]
\end{wideitemize}
\end{frame}

\begin{frame}{Solving Cournot}

The general problem (without specify the cost function or the inverse demand) is:
\[\max_{q_i} P(Q)q_i-C_i(q_i) \]

We take the first-order condition holding fixed the quantity of the other firms:

\[P'(Q)q_i + P(Q)-C'(q_i)=0\]



Although we specified that $N=2$, this is actually the solution for any number of firms! From this we can derive a best-response function for each firm by solving for $q_i$ as a function of all other firm's quantities.

\end{frame}

\begin{frame}{Solving Cournot: Natural Spring Duopoly}

This example follows Example 15.1 from N\&S Ch. 15.
\begin{wideitemize}
    \item Suppose $N=2$.
    \item Suppose $c_i(q_i)=cq_i$, so cost functions are symmetric.
    \item Suppose Inverse demand is $P(Q)=a-Q$, which is a linear demand (remember this from monopoly?)
    \item Suppose the firms compete in quantities. What is the Nash Equilibrium quantities and price?
\end{wideitemize}
For the solution, see handwritten notes.
\end{frame}

\begin{frame}{Adding a Twist: Natural Spring with a Cartel}
\begin{definition}
A \textbf{cartel} is an association of firms that works together to keep prices above the competitive level.
\end{definition}
\begin{wideitemize}
    \item What happens in the previous example when we assume the two firms can perfectly cooperate?
    \item We model perfect cooperation as the two firms acting in unison to maximize total profit.
    \item Find the quantities that would be produced if the firms could form a cartel.
    \item \textbf{Challenge (try on your own).} Compare this quantity to the monopoly quantity.
    \item For the solution see handwritten notes.
\end{wideitemize}

\end{frame}



\begin{frame}{Comparing Three First-Order Conditions}

\textbf{Perfect Competition:} The firm acts as if its actions do not impact price.
\[P(Q)-C_i'(q)=0\]

\textbf{Cournot Oligopoly:} The firm accounts for the fact that it can impact price.

\[P(Q)+P'(Q)q_i -C'(q_i)=0\]

\textbf{Cartel:} The cartel through cooperation has full market power, and accounts for the externality that production of one firm has on the price all other firm's charge.
\[P(Q)+P'(Q)Q-C_i'(q_i)=0\]
Profits can be ordered this way:
\[\pi_{perfect}\leq \pi_{oligpoly}\leq \pi_{cartel}\]

\end{frame}

\begin{frame}{Interpreting Cournot Equilibrium}

\begin{wideitemize}
    \item For any non-infinite number of firms, Cournot/quantity competition results in a price above perfect competition, and thus more profit for firms than under perfect competition.
    \item When $n\to \infty$, it can be shown under general conditions that the price converges to the marginal cost.
    \item As a result, as more firms enter a market, competition increases and we approach perfect competition!
    \item Showing this is an exercise in the practice problems for this week.
\end{wideitemize}
    
\end{frame}

\begin{frame}{Discussion of Cournot vs. Bertrand}

\begin{wideitemize}
    \item In your opinion, do firms compete more in prices or quantities?\pause
    \item In your opinion, which model seems more realistic in terms of interpretation?\pause
    \item Ultimately, the stark differences say something about social science.\pause
    \item Models are tools, but not the truth.\pause
    
\end{wideitemize}
 \vspace{4mm}
     \begin{quote}
        ``All models are wrong, but some are useful.''
        
        - George Box
    \end{quote}
    
\end{frame}
\begin{frame}{Concepts for the Midterm}
The following concepts can potentially be on the midterm:
\begin{wideitemize}
    \item Risk and Uncertainty
    \item Monopoly and Monopsony (if we get to monopsony today)
    \item Static Game Theory (any problem using the tools we discussed)
    \item Oligopoly
\end{wideitemize}

Please submit your practice PDF upload (worth 5 pts) and read the midterm instructions (see week 3). Important instructions:

\begin{enumerate}
    \item Midterm is open book but not open contact.
    \item Midterm will be 1 hour.
    \item Academic dishonesty will not be tolerated.
    \item If you do not do the practice problems, it is highly unlikely you will do well on the test.
    \item Grading is either 40-60\% or 100\% final, whichever results in a better score.
\end{enumerate}

\end{frame}

\begin{frame}{Reminder: Lecture After Midterm}
\centering
 There will be a 1 hour lecture after the midterm. 
 \vspace{7mm}
 
 The content of this lecture will NOT be on the midterm.
 
 \vspace{7mm}
 It will FOR SURE be on the final. Choose attendance strategies accordingly!
    
\end{frame}


\end{document}

