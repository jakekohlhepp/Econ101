%%%%%%%%%%%%%%%%%%%%%%%%%%%%%%%%%%%%%%%%%
% Beamer Presentation
% LaTeX Template
% Version 1.0 (10/11/12)
%
% This template has been downloaded from:
% http://www.LaTeXTemplates.com
%
% License:
% CC BY-NC-SA 3.0 (http://creativecommons.org/licenses/by-nc-sa/3.0/)
%
%%%%%%%%%%%%%%%%%%%%%%%%%%%%%%%%%%%%%%%%%

%----------------------------------------------------------------------------------------
%	PACKAGES AND THEMES
%----------------------------------------------------------------------------------------

\documentclass[aspectratio=169]{beamer}
\usepackage[utf8]{inputenc}
\usepackage{booktabs}
\usepackage{graphicx}
\usepackage{array}
\usepackage{caption}
\usepackage{threeparttable}
\usepackage{lscape}
\usepackage{import}
\usepackage{amsmath}
\usepackage{csvsimple}
\usepackage{siunitx}
\usepackage{subfigure}
\usepackage{filecontents}
\newenvironment{wideitemize}{\itemize\addtolength{\itemsep}{10pt}}{\enditemize}
\usepackage{appendixnumberbeamer}
\usepackage{float}
\usepackage{amsmath}  
\usepackage{tikz,pgfplots}
\usepackage{tkz-fct}
\usepackage{amsthm}
\pgfplotsset{compat=1.10}
\usepgfplotslibrary{fillbetween}
\newcommand{\vertLineFromPoint}[1]{
  \draw[dashed] 
  (#1) -- (#1|-{rel axis cs:0,0})
}
\newcommand{\horLineFromPoint}[1]{
  \draw[dashed] 
  (#1) -- (#1-|{rel axis cs:0,0})
}
\mode<presentation> {
\AtBeginSection[]
{
    \begin{frame}
        \frametitle{Table of Contents}
        \tableofcontents[currentsection]
    \end{frame}
}
% The Beamer class comes with a number of default slide themes
% which change the colors and layouts of slides. Below this is a list
% of all the themes, uncomment each in turn to see what they look like.

%\usetheme{default}
%\usetheme{AnnArbor}
%\usetheme{Antibes} -
%\usetheme{Bergen}
%\usetheme{Berkeley}
%\usetheme{Berlin}
\usetheme{Boadilla}
%\usetheme{CambridgeUS}
%\usetheme{Copenhagen} -
%\usetheme{Darmstadt}
%\usetheme{Dresden}
%\usetheme{Frankfurt}
%\usetheme{Goettingen}
%\usetheme{Hannover}
%\usetheme{Ilmenau}
%\usetheme{JuanLesPins}
%\usetheme{Luebeck}
%\usetheme{Madrid}
%\usetheme{Malmoe}
%\usetheme{Marburg}
%\usetheme{Montpellier}
%\usetheme{PaloAlto}
%\usetheme{Pittsburgh}
%\usetheme{Rochester} -
%\usetheme{Singapore}
%\usetheme{Szeged}
%\usetheme{Warsaw}

% As well as themes, the Beamer class has a number of color themes
% for any slide theme. Uncomment each of these in turn to see how it
% changes the colors of your current slide theme.

%\usecolortheme{albatross}
%\usecolortheme{beaver}
%\usecolortheme{beetle}
%\usecolortheme{crane}
%\usecolortheme{dolphin}
%\usecolortheme{dove}
%\usecolortheme{fly}
%\usecolortheme{lily}
%\usecolortheme{orchid}
%\usecolortheme{rose}
%\usecolortheme{seagull}
%\usecolortheme{seahorse}
%\usecolortheme{whale}
%\usecolortheme{wolverine}

%\setbeamertemplate{footline} % To remove the footer line in all slides uncomment this line
\setbeamertemplate{footline}[frame number] % To replace the footer line in all slides with a simple slide count uncomment this line
\setbeamertemplate{theorems}[numbered]
\setbeamertemplate{navigation symbols}{} % To remove the navigation symbols from the bottom of all slides uncomment this line
}
\setbeamertemplate{caption}{\raggedright\insertcaption\par}
  \setbeamertemplate{enumerate items}[default]
\usepackage{graphicx} % Allows including images
\usepackage{booktabs} % Allows the use of \toprule, \midrule and \bottomrule in tables
%\usepackage {tikz}
\newtheorem*{theorem*}{Theorem}
\newtheorem*{lemma*}{Lemma}
\newtheorem*{proposition}{Proposition}
\newtheorem*{corollary*}{Corollary}
\newtheorem*{definition*}{Definition}
\DeclareMathOperator*{\argmin}{arg\,min}
\newtheorem*{assumption}{Assumption}
\usetikzlibrary {positioning}
%\usepackage {xcolor}

%----------------------------------------------------------------------------------------
%	TITLE PAGE
%----------------------------------------------------------------------------------------

\title[Game Theory]{Lecture 3: Static Game Theory} % The short title appears at the bottom of every slide, the full title is only on the title page

\author{Jacob Kohlhepp} % Your name
\institute[UCLA] % Your institution as it will appear on the bottom of every slide, may be shorthand to save space
{
Econ 101 \\ % Your institution for the title page
\medskip
}
\date{\today} % Date, can be changed to a custom date

\begin{document}

\begin{frame}
\titlepage % Print the title page as the first slide
\end{frame}

\begin{frame}{Introduction}
\begin{wideitemize}
    \item For the rest of this class, we will study the \textbf{strategic interactions} of economic actors.
    \item To do this we need a new tool.
    \item This new tool is called \textbf{game theory.}
\end{wideitemize}

\end{frame}

\begin{frame}{What is Game Theory?}
\begin{definition}
\textbf{Game theory} is the study of mathematical models of strategic interaction among rational decision-makers.\footnote{Myerson (1991)}
\end{definition}
\begin{wideitemize}
    \item Used in many fields: computer science, economics, political science, psychology.
    \item Came to exist as a field around 1928.
    \item Came to prominence in the 1950s with John Nash.
    \item Became a dominant force in economics by the 1970s.
    \item 11 game theorists have won Econ. Nobel Prize, including Shapley (UCLA)
    \item Many legitimate critiques (we will not cover), but still very useful tool.
\end{wideitemize}
\end{frame}

\begin{frame}{Game Theory in This Class}
\begin{wideitemize}
    \item This lecture will cover static game theory: games where actions occur simultaneously in one period.
    \item In a few weeks we will cover dynamic game theory: games where actions can occur sequentially.
    \item The main application of both tools will be \textbf{oligopoly games}.
    \item However the concepts themselves are fair game for the final/midterm.
    \item For example, I may ask you to solve a game that is not oligopoly using Nash Equilibrium/dominant strategies/etc.
\end{wideitemize}

\end{frame}

\begin{frame}{Defining a Game}
A game consists of a list of players, their possible strategies, and the resulting payoffs.
\begin{wideitemize}
    \item \textbf{Players.} A set of decision-makers, assumed to be rational. Indexed by $i=1,...,n$. Often we call the set of players which are not $i$ as $-i$.
    \item \textbf{Strategies.} The set of ``courses of action" available to each player. Denote each strategy as $s_i$, the full set as $S_i$. Denote $s_{-i}$ as the strategies of all players but $i$.
    \begin{wideitemize}
        \item When the game is static this is equivalent to actions.
        \item When the game is not static players may take different actions based on the past actions of others.
    \end{wideitemize}
    \item \textbf{Payoffs.} The utility obtained by all players given the chosen strategies. Denoted by a function $u_i(s_1,...,s_n)$ which maps strategies to a number.
\end{wideitemize}
    
\end{frame}
\begin{frame}{Example: The Prisoner's Dilemma}
We now describe one of the most famous games in game theory. Tell story.
\begin{wideitemize}
    \item \textbf{Players.} Two players, $1,2$.
    \item \textbf{Strategies.} $S_i=\{silent,betray\}$. Each player can either stay silent or betray the other.
    \item \textbf{Payoffs.} 
    \[u_i=\begin{cases} 1 \text{ if } s_i=s_{-i}=silent\\
    3 \text{ if } s_i=betray, \, s_{-i}=silent\\
    0 \text{ if } s_i=silent, \, s_{-i}=betray\\
    1/2 \text{ if } s_i=s_{-i}=betray\\
    \end{cases}\]
\end{wideitemize}

Note: we can usually apply many interpretations to the same game.
\end{frame}

\begin{frame}{Example: The Prisoner's Dilemma}
We can summarize all of this in the normal-form of the game (fill in):

\vspace{6cm}
\end{frame}

\begin{frame}{Solving the Game: Best Response}
\begin{wideitemize}
    \item We can now describe or write down a game.
    \item To understand what is likely to happen when rational players play the game, we need some more concepts.
    \item First idea: What is the best thing for a player to do if that player \textbf{knows} the strategy of the other player?
    \begin{definition}
Strategy $s_i$ is a \textbf{best-response} for player $i$ to other players strategies $s_{-i}$ if:
\[u_i(s_i, s_{-i}) \geq u_i(s_i', s_{-i}) \text{ for all } s_i' \in S_i \]

\end{definition}
\item In words: a best-response is a utility maximizing action given a strategy of others.
\item We denote a best response to the strategy of others $BR_i(s_{-i})$.
\end{wideitemize}


\end{frame}

\begin{frame}{Example: The Prisoner's Dilemma}
Let's find the best-responses in the prisoner's dilemma (fill in)

\vspace{6cm}

It is often helpful to underline the best-response of each player to each other player's action.
\end{frame}

\begin{frame}{Solving the Game: Dominant Strategy}
\begin{wideitemize}
    \item We want to understand what strategies players are \textbf{likely} to play.
    \item We start with a very strong concept called a dominant strategy.
    \begin{definition}
Strategy $s_i$ is a \textbf{dominant strategy} for player $i$ if it is a best-response to all possible combinations of strategies of others.

\end{definition}
\item In words: an action/strategy is dominant if it is best regardless of what others do.
\item Issue: there might not be a dominant strategy.
\item If one does exist, we sense people are likely to play it.
\item Closely related concept: dominated strategy. This is a strategy which yields a lower payoff for every strategy of others than another strategy.
\end{wideitemize}


\end{frame}

\begin{frame}{Example: The Prisoner's Dilemma}
Let's find the dominant strategy of each player in the prisoner's dilemma (fill in). 

\vspace{6cm}


\end{frame}

\begin{frame}{Solving the Game: Nash Equilibrium}
\begin{wideitemize}
    \item We want to understand what outcomes are likely to occur.
    \item We now use a concept of equilibrium, developed by Nash.
    \item This will be the main concept we use in this class for solving static games.
    \begin{definition}
A \textbf{Nash equilibrium} is a strategy for every player $(s_1^*,...,s_n^*)$ such that for each player $i=1,2,..,n$ $s_i^*$ is a best-response to the strategies of others $s_{-i}^*$

\end{definition}
\item In words: everyone is playing a strategy which is a best-response to everyone else's strategy.
\item Intuition: if we gave any individual player the option to deviate while holding everyone else's action fixed, no one would deviate.
\end{wideitemize}


\end{frame}

\begin{frame}{Example: The Prisoner's Dilemma}
Let's find a Nash Equilibrium in the prisoner's dilemma (fill in).


\vspace{6cm}

\end{frame}

\begin{frame}{Comments on Nash Equilibrium}
\begin{wideitemize}
    \item With finite players and finite actions, Nash proved that every game has at least one Nash equilibrium.
    \item However sometimes the only NE is in mixed strategies (more on this next).
    \item If there is a dominant strategy for a player, it must be part of the NE.
    \item If every player has a dominant strategy, then there is a unique Nash Equilibrium.
    %\item Actually if we iteratively remove strictly dominated strategies and we find a unique strategy at the end this is the unique Nash Equilibrium. 

\end{wideitemize}


\end{frame}

\begin{frame}{New Example: Penalty Kick}
Consider the following situation:
\begin{wideitemize}
    \item \textbf{Players:} Goalkeeper trying to defend goal, kicker trying to score.
    \item \textbf{Actions:} $S_g=\{kick left, kick right\}$, $S_k=\{dive left, dive right\}$.
    \item \textbf{Payoffs:} Goalkeeper gets 1 if dives same way as kick, 0 if not. Kicker gets 1 if kicks the way the goalkeeper does not dive, 0 if not.
    
\end{wideitemize}

Write down the normal form of this game with payoffs yourself. Check for a Nash Equilibrium in pure strategies.

\end{frame}

\begin{frame}{Pure Strategy Nash Equilibrium of Penalty Kick Game}

(fill in) 

\vspace{6cm}


\end{frame}



\begin{frame}{Solving a Game: Mixed Nash Equilibrium}
\begin{wideitemize}
    \item We observe that in real soccer games the goalkeeper must essentially try and predict which way the ball gets kicked.
    \item Knowing this, kickers tend to randomize. If they are predictable, this will make them score less.
    \item This is a mixed strategy Nash Equilibrium: players are playing random combinations of pure strategies. Example: $1/3$ left, $2/3$ right.
    \begin{definition}
    A \textbf{mixed strategy} is a probability distribution over the possible actions.
    \end{definition}
    \item In words: Mixed strategies assign probabilities to each action. A pure strategy is just a mixed strategy where one action has probability 1 and all others have probability 0.
    \item A \textbf{mixed strategy Nash Equilibrium} is then an NE where players are playing mixed strategies (probability not 1 or 0 for some actions). 
\end{wideitemize}
\end{frame}

\begin{frame}{New Example: Penalty Kick}

Let's find the mixed strategy Nash Equilibrium of the penalty kick game. See handwritten notes. Please take it easy on yourself if this is hard (fill in).

\vspace{5cm}

\textbf{Hint:} A trick to find mixed NE is that when players are playing strictly mixed strategies players must be indifferent between each pure action. This gives an \textbf{indifference condition} for each player which we can use to solve for the probabilities.
\end{frame}


\begin{frame}{Continuous Actions}

\begin{wideitemize}
    \item So far we have focuses on discrete actions.
    \item Often we are concerned with continuous choices, like setting prices and quantities.
    \item Most of the same concepts apply to continuous actions, we just need to use calculus to get best-responses.
\end{wideitemize}
    
\end{frame}

\begin{frame}{Example: Lawn Maintenance (Positive Externalities)}
\begin{wideitemize}
    \item \textbf{Players:} 2 neighbors ($i=1,2$)
    \item \textbf{Actions:} Number of hours to spend on lawn maintenance: $l_i\geq 0$. Note this is a continuous choice, so there are infinite possible strategies.
    \item \textbf{Payoffs:} The benefit per hour is given by:
    \[10-l_i + \frac{l_{-i}}{2}\]
    the cost is 4 per hour.
\end{wideitemize}

\hfill Source: N\&S 8.4

Write the complete payoff function yourself (fill in).

\vspace{2cm}
\end{frame}

\begin{frame}{Solving Lawn Maintenance}
We now find the best-response functions and the Nash equilibrium (fill in).

\vspace{6cm}

\end{frame}
\begin{frame}{Interpretation of Lawn Maintenance game}

\begin{wideitemize}
    \item Question: Is the Nash Equilibrium socially optimal or Pareto efficient? Equivalently: is there a number of lawn maintenance hours that would leave both strictly better off?
    \item The answer is yes. But how do we show it formally?
    \item One way is to maximize the sum of their utilities.
    \item This is equivalent to asking what lawn maintenance hours a benevolent social planner would choose, where benevolent is one who cares about the sum of utility.

\end{wideitemize}
    
\end{frame}

\begin{frame}{Maximizing the Sum of Utility: A Benevolent Social Planner}

To implement this, we solve:
\[\max_{l_1, l_2} u_1(l_1, l_2) + u_2(l_2, l_1) = \max_{l_1, l_2}(10-l_1 + \frac{l_{2}}{2})l_1 - 4l_1 + (10-l_2 + \frac{l_{1}}{2})l_2 - 4l_2 \]

Call total utility $U$. Then take FOCs:
\[\frac{\partial U}{\partial l_1} = (10-l_1 + \frac{l_{2}}{2}) -l_1 -4 +l_2/2 =0 \]

\[\frac{\partial U}{\partial l_2} = (10-l_2 + \frac{l_{1}}{2}) -l_2 -4 +l_1/2 =0 \]

Notice these FOCs account for the externality!
\end{frame}

\begin{frame}{Maximizing the Sum of Utility: A Benevolent Social Planner}

Continuing the work, the FOCs can be re-written as:
\[(6 + l_2)/2 =l_1 \]
\[(6 + l_1)/2 =l_2 \]

Which we can solve to find:
\[l_1^* = \bigg ( 6 + (6 + l_1^*)/2 \bigg )/2 \implies l_1^* = 6\]
By symmetry:
\[l_2^* = 6\]

This is a pareto improvement, because $u_1(6,6) = u_2(6,6)=18$ which is greater than $u_1(4,4) = u_2(4,4)=16$. So it is a strict Pareto improvement over the Nash equilibrium!


\end{frame}
\begin{frame}{Game Theory Concepts To Know for the Midterm/Final}

\begin{wideitemize}
    \item Dominant Strategies/Dominated Strategy
    \item Best-Response
    \item Nash Equilibrium/Mixed NE
    \item Continuous Action Best-Response/Nash Equilibrium
\end{wideitemize}
    
\end{frame}

\end{document}

