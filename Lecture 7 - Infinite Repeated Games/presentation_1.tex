%%%%%%%%%%%%%%%%%%%%%%%%%%%%%%%%%%%%%%%%%
% Beamer Presentation
% LaTeX Template
% Version 1.0 (10/11/12)
%
% This template has been downloaded from:
% http://www.LaTeXTemplates.com
%
% License:
% CC BY-NC-SA 3.0 (http://creativecommons.org/licenses/by-nc-sa/3.0/)
%
%%%%%%%%%%%%%%%%%%%%%%%%%%%%%%%%%%%%%%%%%

%----------------------------------------------------------------------------------------
%	PACKAGES AND THEMES
%----------------------------------------------------------------------------------------

\documentclass[aspectratio=169]{beamer}
\usepackage[utf8]{inputenc}
\usepackage{booktabs}
\usepackage{graphicx}
\usepackage{array}
\usepackage{caption}
\usepackage{threeparttable}
\usepackage{lscape}
\usepackage{import}
\usepackage{multirow, array}
\usepackage{amsmath}
\usepackage{csvsimple}
\usepackage{siunitx}
\usepackage{subfigure}
\usepackage{filecontents}
\newenvironment{wideitemize}{\itemize\addtolength{\itemsep}{10pt}}{\enditemize}
\usepackage{appendixnumberbeamer}
\usepackage{float}
\usepackage{amsmath}  
\usepackage{tikz,pgfplots}
\usepackage{tkz-fct}
\usepackage{amsthm}
\pgfplotsset{compat=1.10}
\usepgfplotslibrary{fillbetween}
\newcommand{\vertLineFromPoint}[1]{
  \draw[dashed] 
  (#1) -- (#1|-{rel axis cs:0,0})
}
\newcommand{\horLineFromPoint}[1]{
  \draw[dashed] 
  (#1) -- (#1-|{rel axis cs:0,0})
}
\mode<presentation> {
\AtBeginSection[]
{
    \begin{frame}
        \frametitle{Table of Contents}
        \tableofcontents[currentsection]
    \end{frame}
}

% The Beamer class comes with a number of default slide themes
% which change the colors and layouts of slides. Below this is a list
% of all the themes, uncomment each in turn to see what they look like.

%\usetheme{default}
%\usetheme{AnnArbor}
%\usetheme{Antibes} -
%\usetheme{Bergen}
%\usetheme{Berkeley}
%\usetheme{Berlin}
\usetheme{Boadilla}
%\usetheme{CambridgeUS}
%\usetheme{Copenhagen} -
%\usetheme{Darmstadt}
%\usetheme{Dresden}
%\usetheme{Frankfurt}
%\usetheme{Goettingen}
%\usetheme{Hannover}
%\usetheme{Ilmenau}
%\usetheme{JuanLesPins}
%\usetheme{Luebeck}
%\usetheme{Madrid}
%\usetheme{Malmoe}
%\usetheme{Marburg}
%\usetheme{Montpellier}
%\usetheme{PaloAlto}
%\usetheme{Pittsburgh}
%\usetheme{Rochester} -
%\usetheme{Singapore}
%\usetheme{Szeged}
%\usetheme{Warsaw}

% As well as themes, the Beamer class has a number of color themes
% for any slide theme. Uncomment each of these in turn to see how it
% changes the colors of your current slide theme.

%\usecolortheme{albatross}
%\usecolortheme{beaver}
%\usecolortheme{beetle}
%\usecolortheme{crane}
%\usecolortheme{dolphin}
%\usecolortheme{dove}
%\usecolortheme{fly}
%\usecolortheme{lily}
%\usecolortheme{orchid}
%\usecolortheme{rose}
%\usecolortheme{seagull}
%\usecolortheme{seahorse}
%\usecolortheme{whale}
%\usecolortheme{wolverine}

%\setbeamertemplate{footline} % To remove the footer line in all slides uncomment this line
\setbeamertemplate{footline}[frame number] % To replace the footer line in all slides with a simple slide count uncomment this line
\setbeamertemplate{theorems}[numbered]
\setbeamertemplate{navigation symbols}{} % To remove the navigation symbols from the bottom of all slides uncomment this line
}
\setbeamertemplate{caption}{\raggedright\insertcaption\par}
  \setbeamertemplate{enumerate items}[default]
\usepackage{graphicx} % Allows including images
\usepackage{booktabs} % Allows the use of \toprule, \midrule and \bottomrule in tables
%\usepackage {tikz}
\newtheorem*{theorem*}{Theorem}
\newtheorem*{lemma*}{Lemma}
\newtheorem*{proposition}{Proposition}
\newtheorem*{corollary*}{Corollary}
\newtheorem*{definition*}{Definition}
\DeclareMathOperator*{\argmin}{arg\,min}
\newtheorem*{assumption}{Assumption}
\usetikzlibrary {positioning}
% macro for inputting terminal nodes
\newcommand\term[2]{\node[below]at(#1){$#2$};}
%
%\usepackage {xcolor}

%----------------------------------------------------------------------------------------
%	TITLE PAGE
%----------------------------------------------------------------------------------------

\title[Infinite]{Lecture 7: Infinite Repeated Games} % The short title appears at the bottom of every slide, the full title is only on the title page

\author{Jacob Kohlhepp} % Your name
\institute[UCLA] % Your institution as it will appear on the bottom of every slide, may be shorthand to save space
{
Econ 101 \\ % Your institution for the title page
\medskip
}
\date{\today} % Date, can be changed to a custom date

\begin{document}

\begin{frame}
\titlepage % Print the title page as the first slide
\end{frame}

\begin{frame}{Introduction}
\begin{wideitemize}
    \item Last Lecture: sequential play introduces interesting forces in economic models.
    \item Today: what if the same game is played infinitely many times?
    \item Because players know they will always see each other again, this introduces new forces.

\end{wideitemize}
\end{frame}

\begin{frame}{Applications}
\begin{wideitemize}
    \item How do OPEC countries sustain high oil prices (above the Cournot oligopoly level).
    \item How do employers motivate employees without performance pay?
    \item How do people sustain economic relationships in situations or places where there is no rule of law?
    \item How do we model reputation or relationships?
\end{wideitemize}
    
\end{frame}

\begin{frame}{Some Additional Tools: Discounting}

\begin{wideitemize}
    \item Main solution concept remains \textbf{subgame-perfect Nash equilibrium.}
    \item But because payoffs are infinite sums, they will be unbounded unless they are discounted.
    \begin{definition}
    The discount factor represents the value of 1 util tomorrow if it is provided today. It is always between 0 and 1, and with it we can write the present utility from a stream of utilities $\{u_t\}$ as:
    \[U = \sum^\infty_{t=0} \delta^t u_t = u_0 + \delta u_1+\delta^2 u_2+...\]
    \end{definition}
    \item Interpretations:
    \begin{enumerate}
        \item The time value of money means that cash today is more valuable than cash tomorrow because we can always put it into a savings account. Therefore the interest rate drives the discount rate.
        \item $\delta$ can represent the probability players meet again (uncertainty about the future).
    \end{enumerate}
\end{wideitemize}
    
\end{frame}

\begin{frame}{Some Additional Tools: Infinite Sums}
Suppose we want to evaluate $\sum^\infty_{t=0} \delta^t u $, where $u$ is some constant payoff. It is useful to remember that:

\[\sum^\infty_{t=0} \delta^t u = \frac{u}{1-\delta} \]

This will help us calculate the utility of receiving a payoff of $u$ every period forever, including today. If we want to evaluate the payoff of recieving $u$ every period forever (except for today) that would be:

\[\sum^\infty_{t=1} \delta^t u = \frac{\delta u}{1-\delta} \]
    
\end{frame}

\begin{frame}{Some Additional Tools: Stage Game }
    \begin{definition}
    The \textbf{stage game} is the subgame that is played each time period.
    \end{definition}
    
   Suppose we consider a situation where two players play the prisoner's dilemma game every period for infinitely many periods. The stage game looks like:
    \begin{table}
    \setlength{\extrarowheight}{2pt}
    \begin{tabular}{cc|c|c|}
      & \multicolumn{1}{c}{} & \multicolumn{2}{c}{Player $1$}\\
      & \multicolumn{1}{c}{} & \multicolumn{1}{c}{$Silent$}  & \multicolumn{1}{c}{$Betray$} \\\cline{3-4}
      \multirow{2}*{Player $2$}  & $Silent$ & $(2,2)$ & $(0,3)$ \\\cline{3-4}
      & $Betray$ & $(3,0)$ & $(1,1)$ \\\cline{3-4}
    \end{tabular}
  \end{table}
  where the only NE is $(Betray, Betray)$ in the stage game.
\end{frame}

\begin{frame}{Considering Finitely Repeated Prisoner's Dilemma}
    Suppose 2 players player prisoner's dilemma T times where $1<T<\infty$. What is the subgame perfect Nash equilibrium? Hint: Let's use backwards induction.
\end{frame}
\begin{frame}{Some Additional Tools: Trigger Strategies}

\begin{wideitemize}
    \item Now we wish to find an SPNE of the repeated prisoner's dilemma. To do this we will consider the following types of strategies.
    
    \begin{definition}
A \textbf{trigger strategy} is one where a player initially cooperates until the other player deviates from cooperation, after which the player punishes the other player for a certain period of time.
\end{definition}

\item The most severe types of trigger strategies are called grim trigger strategies: they involve cooperating until one player defects. Then both players ``punish" each other for the rest of eternity.

\item ``Punishment" usually refers to playing one of the static Nash Equilibria.
\item In the case of the prisoner's dilemma it involves playing $(Betray, Betray)$.

\end{wideitemize}

\end{frame}

\begin{frame}{Solving the Infinitely Repeated Prisoner's Dilemma}
Suppose two players with the same discount factor $\delta$ play the prisoner's dilemma infinitely many times.
\begin{wideitemize}
    \item We wish to find an SPNE of the game where cooperation is sustained.
    \item In PD, cooperation refers to sustaining the strategy $(Silent, Silent)$ which yields a higher payoff than the static NE of $(Betray, Betray)$
    \item To do this, we need to come up with a strategy that supports $(Silent, Silent)$.
    \item Let's try grim trigger: each player plays $Silent$ until they observe the other player $Betray$.
    \item After the first $Betray$ both players play $Betray$ forever.
\end{wideitemize}
    
\end{frame}

\begin{frame}{Solving the Infinitely Repeated Prisoner's Dilemma}
\begin{wideitemize}
    \item We now need to check that grim trigger is indeed a PD.
    \item Two conditions:
    \begin{enumerate}
        \item No player wants to Betray given no betrayal has occurred yet.
        \item If a betrayal occurs, no player wants to play $Silent$.
    \end{enumerate}
    \item These are the only requirements for SPNE because of the following principle.
    \begin{definition}
    The \textbf{one-shot deviation principle} states that a strategy profile is a subgame-perfect Nash equilibrium if and only if no player can increase their payoff by changing a single decision in a single period. 
    \end{definition}
    \item Because players are symmetric in the prisoner's dilemma we only need to check for one player. 
    \item There are two conditions because there are only two distinct subgames.
    
\end{wideitemize}
\end{frame}

\begin{frame}{Solving the Infinitely Repeated Prisoner's Dilemma}

\begin{wideitemize}
    \item We will now derive the two inequalities that these two conditions imply, and check when they are satisfied. (See handwritten notes). \pause
    \item It turns out that these two conditions reduce to just one condition, which is that the gain from cooperating (future payoffs) is greater than the one-time gain from betrayal.
    \item This holds whenever $\delta > 1/2$, that is whenever both players are sufficiently patient or forward looking.
    \item This is surprising, because in any finite repeated prisoner's dilemma game (when we repeat the game $T<\infty$ times) the only SPNE is to both betray. (It is a good exercise to derive this yourself).
    
\end{wideitemize}


\end{frame}

\begin{frame}{Folk Theorem}

The finding from the prisoner's dilemma, that we can support most outcomes in an infinitely repeated game given players are sufficiently patient, applies to many other games!
\begin{theorem}
The \textbf{Folk Theorem} states that in an infinitely repeated game with discounting, we can support any strategy which has strictly higher payoff than the minmax payoff\footnote{minmax payoff is the maximum payoff a player can garuntee him/herself given the other player is trying to give them the lowest payoff.} if $\delta$ is close enough to 1.
\end{theorem}
Intuition: With enough patience, virtually anything is possible with an infinite future.

\end{frame}


\begin{frame}{Comments}

When I took this class as an undergraduate, this section of the class felt like dark magic. I hope these comments prove helpful.

\begin{wideitemize}
    \item Grim trigger is not the only way to sustain cooperation. There are many others, including tit-for-tat (see problem set).
    \item Infinitely repeated games are hard to understand because they infinite sequences of occurrences.
    \item More complicated infinitely repeated games that involve signals and other things are actual the cutting edge of economic theory.
    \item Some of the best uses of these repeated games are in political science.
    \item Grim trigger is a useful way to understand the Cold War policy of mutually assured destruction (MAD).
\end{wideitemize}
    
\end{frame}
\begin{frame}{Main Application: Collusion in Quantities}
\begin{wideitemize}
    \item $N$ oil producing countries with common discount factor $\delta$ and unit production cost of $c=10$.
    \item $t=1,2,...$ (infinite periods).
    \item Every period the countries simultaneously compete in quantities (they play a Cournot stage game).
    \item Inverse market demand is given by $p = 50-\frac{Q}{100} $
    \item Find the stage game NE, and the single period profit from this strategy.
    \item Find the maximum achievable profit and the quantity that achieves it.
    \item Use grim trigger with the stage NE as a punishment to support the maximum achievable profit.
\end{wideitemize}
See handwritten notes for solution.
\end{frame}

\begin{frame}{Collusion Solution  }

\begin{wideitemize}
    \item The condition needed to support the monopoly quantity is:
    
    \item Notice that this inequality becomes harder to support when $N$ (the number of countries) rises and when $\delta $ falls.
    \item Interpretation: collusion is easier when there are fewer players and players are more patient.
    \item N\&S call this \textbf{tacit collusion} because there is no explicit association between countries - repeated interaction organically supports collusion.
    \item I prefer to think of this as a relational cartel: oil producing countries forming an alliance based on future profits.
    \item This actually occurs in real life! OPEC keeps prices artificially high by jointly restricting supply.
\end{wideitemize}
    
\end{frame}
\end{document}

