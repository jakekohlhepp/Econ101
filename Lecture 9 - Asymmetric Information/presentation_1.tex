%%%%%%%%%%%%%%%%%%%%%%%%%%%%%%%%%%%%%%%%%
% Beamer Presentation
% LaTeX Template
% Version 1.0 (10/11/12)
%
% This template has been downloaded from:
% http://www.LaTeXTemplates.com
%
% License:
% CC BY-NC-SA 3.0 (http://creativecommons.org/licenses/by-nc-sa/3.0/)
%
%%%%%%%%%%%%%%%%%%%%%%%%%%%%%%%%%%%%%%%%%

%----------------------------------------------------------------------------------------
%	PACKAGES AND THEMES
%----------------------------------------------------------------------------------------

\documentclass[aspectratio=169]{beamer}
\usepackage[utf8]{inputenc}
\usepackage{booktabs}
\usepackage{graphicx}
\usepackage{array}
\usepackage{caption}
\usepackage{threeparttable}
\usepackage{lscape}
\usepackage{import}
\usepackage{multirow, array}
\usepackage{amsmath}
\usepackage{csvsimple}
\usepackage{siunitx}
\usepackage{subfigure}
\usepackage{filecontents}
\newenvironment{wideitemize}{\itemize\addtolength{\itemsep}{10pt}}{\enditemize}
\usepackage{appendixnumberbeamer}
\usepackage{float}
\usepackage{amsmath}  
\usepackage{tikz,pgfplots}
\usepackage{tkz-fct}
\usepackage{amsthm}
\pgfplotsset{compat=1.10}
\usepgfplotslibrary{fillbetween}
\newcommand{\vertLineFromPoint}[1]{
  \draw[dashed] 
  (#1) -- (#1|-{rel axis cs:0,0})
}
\newcommand{\horLineFromPoint}[1]{
  \draw[dashed] 
  (#1) -- (#1-|{rel axis cs:0,0})
}
\mode<presentation> {
\AtBeginSection[]
{
    \begin{frame}
        \frametitle{Table of Contents}
        \tableofcontents[currentsection]
    \end{frame}
}

% The Beamer class comes with a number of default slide themes
% which change the colors and layouts of slides. Below this is a list
% of all the themes, uncomment each in turn to see what they look like.

%\usetheme{default}
%\usetheme{AnnArbor}
%\usetheme{Antibes} -
%\usetheme{Bergen}
%\usetheme{Berkeley}
%\usetheme{Berlin}
\usetheme{Boadilla}
%\usetheme{CambridgeUS}
%\usetheme{Copenhagen} -
%\usetheme{Darmstadt}
%\usetheme{Dresden}
%\usetheme{Frankfurt}
%\usetheme{Goettingen}
%\usetheme{Hannover}
%\usetheme{Ilmenau}
%\usetheme{JuanLesPins}
%\usetheme{Luebeck}
%\usetheme{Madrid}
%\usetheme{Malmoe}
%\usetheme{Marburg}
%\usetheme{Montpellier}
%\usetheme{PaloAlto}
%\usetheme{Pittsburgh}
%\usetheme{Rochester} -
%\usetheme{Singapore}
%\usetheme{Szeged}
%\usetheme{Warsaw}

% As well as themes, the Beamer class has a number of color themes
% for any slide theme. Uncomment each of these in turn to see how it
% changes the colors of your current slide theme.

%\usecolortheme{albatross}
%\usecolortheme{beaver}
%\usecolortheme{beetle}
%\usecolortheme{crane}
%\usecolortheme{dolphin}
%\usecolortheme{dove}
%\usecolortheme{fly}
%\usecolortheme{lily}
%\usecolortheme{orchid}
%\usecolortheme{rose}
%\usecolortheme{seagull}
%\usecolortheme{seahorse}
%\usecolortheme{whale}
%\usecolortheme{wolverine}

%\setbeamertemplate{footline} % To remove the footer line in all slides uncomment this line
\setbeamertemplate{footline}[frame number] % To replace the footer line in all slides with a simple slide count uncomment this line
\setbeamertemplate{theorems}[numbered]
\setbeamertemplate{navigation symbols}{} % To remove the navigation symbols from the bottom of all slides uncomment this line
}
\setbeamertemplate{caption}{\raggedright\insertcaption\par}
  \setbeamertemplate{enumerate items}[default]
\usepackage{graphicx} % Allows including images
\usepackage{booktabs} % Allows the use of \toprule, \midrule and \bottomrule in tables
%\usepackage {tikz}
\newtheorem*{theorem*}{Theorem}
\newtheorem*{lemma*}{Lemma}
\newtheorem*{proposition}{Proposition}
\newtheorem*{corollary*}{Corollary}
\newtheorem*{definition*}{Definition}
\DeclareMathOperator*{\argmin}{arg\,min}
\newtheorem*{assumption}{Assumption}
\usetikzlibrary {positioning}
% macro for inputting terminal nodes
\newcommand\term[2]{\node[below]at(#1){$#2$};}
%
%\usepackage {xcolor}

%----------------------------------------------------------------------------------------
%	TITLE PAGE
%----------------------------------------------------------------------------------------

\title[Asymmetric]{Lecture 9: Asymmetric Information} % The short title appears at the bottom of every slide, the full title is only on the title page

\author{Jacob Kohlhepp} % Your name
\institute[UCLA] % Your institution as it will appear on the bottom of every slide, may be shorthand to save space
{
Econ 101 \\ % Your institution for the title page
\medskip
}
\date{\today} % Date, can be changed to a custom date

\begin{document}

\begin{frame}
\titlepage % Print the title page as the first slide
\end{frame}

\begin{frame}{Introduction}
\begin{wideitemize}
    \item Last time: introduction to the study of games with incomplete information.
    \item Generally we are interested in games where one party knows more then the other.
    \item Situations that exhibit this property are said to have asymmetric information.
    \item Asymmetric information leads to a failure of the first welfare theorem: equilibria are often not Pareto efficient.
    \item \textbf{Hidden Action: } one player can take an action that is unobserved by the other.
    \begin{wideitemize}
        \item Example: An employee working remotely can exert more or less effort.
    \end{wideitemize}
    \item \textbf{Hidden Type:} one player knows something about the game that the other does not.
    \begin{wideitemize}
        \item Example: A used-car salesman knows the true condition of a car.
    \end{wideitemize}
\end{wideitemize}
\end{frame}

\section{Hidden Action}

\begin{frame}{Managers and Shareholders}

\begin{wideitemize}
    \item One manager and one representative shareholder.
    \item Profit is random but depends on manager effort (e):$\pi \sim N(e,\sigma^2) $
    \item Shareholders are risk neutral and their payoff is $\pi $ minus any payments to the manager.
    \item The manager is risk averse with exponential utility from money (x) given by $ \frac{1-e^{-\theta x}}{\theta} $. Effort has cost $c(e)$ which is increasing and convex.
    \item Thus overall utility to the manager is:
    \[U(x,e) = E \bigg [ \frac{1-e^{-\theta x}}{\theta} \bigg ]  - c(e)\]
    \item If the manager rejects contract both players get 0.
    \item Payments are always of the form $a+b\cdot x$ where $x$ can be effort or profit or something else.
    \item Timing: shareholder proposes contract, manager accepts or rejects, manager exerts effort.
\end{wideitemize}
    
\end{frame}

\begin{frame}{Manager and Shareholders: Asymmetric Information, Hidden Action}
Before proceeding, let's take a moment to re-write utility without expectations.
\begin{wideitemize}
    \item For any fixed $a,b$ the payment to the manager is a random variable:
    \[a+b\pi = a+bN(e, \sigma^2)=N(a+be, b^2\sigma^2)\]
    \item Remember lecture 1:  exponential utility, normal RV means utility can be written as mean-variance form:
    \[E[x] - \theta Var(x)/2\]
    \item Plug in mean and variance:
    \[a+be - \theta b^2\sigma^2/2\]
    \item Remember to subtract the cost of effort:
    \[U(e)=a+be - \theta b^2\sigma^2/2-c(e)\]
    \item See handwritten notes for full solution!
\end{wideitemize}
\end{frame}


\begin{frame}{Manager and Shareholders: Perfect Info, No Hidden Action}
Suppose effort is perfectly observable and the shareholders can pay the manager based on it instead of based on profit.

\begin{wideitemize}
    \item Then the shareholders will always pay based on effort NOT profit. (Why?)
    \item So the payment to the manager is of the form $a+b\cdot e$ (a bonus based on effort).
    \item There is no uncertainty, so we use SPNE.
    \item See handwritten notes for solution. \pause 
    \item Solution:
    \[c'(e^*)=1\]
\end{wideitemize}
    
\end{frame}

\begin{frame}{Manager and Shareholders: Asymmetric Information, Hidden Action}
Now suppose the shareholders only see profit not effort.
\begin{wideitemize}
    \item The payment is of the form $a+b\pi $.
    \item Notice that if $b=0$, $e=0$.
    \item So to get effort we need positive $b$.
    \item But positive $b$ imposes risk on a risk averse manager! So there is a tradeoff.
\end{wideitemize}
    
\end{frame}



\begin{frame}{Manager and Shareholders: Solution}
\begin{wideitemize}
    \item The solution with hidden action:
\[c'(e^{**}) = \frac{1}{1+\theta \sigma^2 c''(e^{**})}\]
\item Note that $\theta \sigma^2 c''(e^{**})>0$ so $c'(e^**)<1$.
\item Less effort with hidden action!
\item We can think of $b$ as the ``power of incentives"
\item In equilibrium $b^{**}=c'(e^{**})$, so the power of incentives decrease with risk aversion and with variance.
\item Main interpretation: there is a trade-off between risk and incentives.
\item Higher $b$ means more effort but also more risk.

\end{wideitemize}

    
\end{frame}

\begin{frame}{Hidden Action Comments}

\begin{wideitemize}
    \item The fact that effort is under-provided in equilibrium is an example of market failure when information is asymmetric.
    \item When risk attitudes cause inefficient effort provision we say there is moral hazard.
    \item The term moral hazard has a more specific meaning in the insurance context:
    
    \begin{definition}
    \textbf{Morla hazard in insurance } is when an economic actor inefficiently chooses not to mitigate risk because another economic actor bears the cost of that risk.
    \end{definition}
    \item  Can you see how health or car insurance could cause this? See the problem set to see this fully worked out!
    \item We will discuss how this impacted Obamacare next lecture.
\end{wideitemize}
    
\end{frame}

\section{Hidden Type}

\begin{frame}{The Market for Lemons\footnote{Aklerlof, Nobel Prize winner.}}

\begin{wideitemize}
    \item Car quality $q$ is uniform between $[0,20000]$.
    \item A buyer who wants to buy a car but only knows the distribution of used car values (not the value of this particular car).
    \item Buyer utility from a car of quality $q$ is $q+b$ less the price.
    \item Seller utility from selling a car is the price. Utility from not selling is $q$.
    \item Consider a sequential game: buyer sets price $p$. Then seller decides whether to sell.
\end{wideitemize}
\end{frame}

\begin{frame}{The Market for Lemons: Full Information}
\begin{wideitemize}
    \item  Suppose before the start of the game quality is fully observed by both buyer and seller.
    \item What happens?\pause
    \item Backwards induct: seller sells if $p\geq q$.\pause
    \item Knowing this, buyer sets minimum price satisfying the inequality, so $p=q$.
    \item All cars are sold! What is surplus?\pause
    \item Seller utility is $q$, which is exactly outside option. Buyer utility is the full gains from trade:
    \[u=q+b-p=b\]
\end{wideitemize}

\end{frame}

\begin{frame}{The Market for Lemons: Hidden Type}
\begin{wideitemize}
    \item Suppose now that only the seller knows the quality of the car.
    \item The buyer only knows the distribution of quality.
    \item But the buyer knows the seller knows the true quality.
\end{wideitemize}
    
\end{frame}


\begin{frame}{The Market for Lemons: Worked Out Solution}
    \begin{wideitemize}
        \item Begin by observing that sellers only sell if $p\geq q$.
        \item So buyer utility from price $p$ is:
        \[\int^p_0 (q+b-p)\frac{1}{20000}dq + \int_p^{20000} 0 \cdot \frac{1}{20000} dq = ((b-p)q +q^2/2|^p_0 = bp-p^2/2  \]
        \item Buyer maximizes this with respect to price which gives FOC:
        \[\frac{dU}{dp} = b - p =0 \implies b^*=p\]
        \item So Perfect Bayesian Equilibrium is buyer sets price $p$, sellers sell if $p\geq q$.
    \end{wideitemize}
\end{frame}
\begin{frame}{The Market for Lemons: Hidden Type Solution}

\begin{wideitemize}
    \item The optimal price is given by:
    \[p^* = b\]
    \item Then the cars that are sold are those with $q\in [0,b]$.
    \item Only the worst cars sell: there is adverse selection (more on this soon).
    \item As the gains from trade ($b$) fall the probability of a transactions shrinks to 0.
    \item Notice that if the seller set the price this game would be much more complicated! 
    \item We would need to consider how the buyer updates their belief about quality after seeing the price the seller posts.
\end{wideitemize}
    
\end{frame}

\begin{frame}{Comments on Hidden Types}
\begin{wideitemize}
    \item In our example, the inability of the buyer to know the type of the seller leads to market failure.
    \begin{definition}
    An economic interaction exhibits \textbf{adverse selection} if one side has more information than the other side, and the informed party uses this information to selectively participate in trades in a way that reduces the utility of the other party.
    \end{definition}
    \item Example: Spence job-market, ``Ban the Box," can you think of others?
    \item This is an especially prominent issue with insurance. See the recommended problem set!
    \begin{definition}
    \textbf{Adverse selection in insurance} refers to the phenomenon that risky types are more likely to accept insurance policies and are also more expensive to serve.
    \end{definition}
    \item This was a main concern with Obamacare! (stay tuned for next lecture)
\end{wideitemize}

    
\end{frame}
\end{document}

